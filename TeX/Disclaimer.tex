% vim: ft=tex

\begin{center}
    { \large
        \begin{minipage}[h]{0.7\textwidth}
            \begin{tcolorbox}[colback=red!7,colframe=red!70!black,fonttitle=\bfseries,title=Disclaimer]
                Dieses \textit{Chordbook} stellt eine pers\"{o}nliche und vielf\"{a}ltige Sammlung von Liedgut dar, die ich \"{u}ber die Jahre gesammelt habe.
                Ich erhebe keinen Anspruch auf Eigentum oder Richtigkeit der Dargestellten Texte oder Akkorde, z.T. habe ich sie von anderen Menschen \"{u}bernommen und mit der Zeit selbst abgewandelt, z.T. habe ich sie mir selber durch H\"{o}ren oder Video-Material abkupfern k\"{o}nnen. \par\medskip

                Sollte es Einw\"{a}nde gegen ein oder mehrere Lieder oder Texte geben, bin ich \"{u}ber meine E-Mail Adresse \href{mailto:chordbook@barzh.de}{\texttt{chordbook@barzh.de}} erreichbar. \par\medskip

                Dank \textsc{Fagus vom Friedrichshain} konnte ich mein Chordbook um einige St\"{u}cke erweitern.
                Seine Beitr\"{a}ge sind gekennzeichnet oder werden noch gekennzeichnet werden.
            \end{tcolorbox}
        \end{minipage}
    }

            \vfill

    { \large
        \begin{minipage}[h]{0.7\textwidth}
            \begin{tcolorbox}[colback=red!7,colframe=red!70!black,fonttitle=\bfseries,title=Stimmung der Gitarre]
                Meine Gitarre ist in \texttt{D} gestimmt \texttt{(DGCFAd)}, d.h. die aufgeführten Akkorde sind nur "`Griffmuster"'.
                Ggf. aufgeführte \texttt{capo x} Ansagen beziehen die tiefer gestimmte Gitarre mit ein.
                Eine Stück mit \texttt{capo 4} auf meiner Gitarre ist also ein \texttt{capo 2} auf "`normalen"' Gitarren.
            \end{tcolorbox}
        \end{minipage}
    }
\end{center}
