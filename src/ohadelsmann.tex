\stepcounter{subsection}

\beginsong{Oh Adelsmann}[%
    by={Versengold},
    index={ohadelsmann}]

    \phantomsection
    \addcontentsline{toc}{subsection}{\thesubsection \quad Oh Adelsmann - Versengold}

    \label{ohadelsmann}

    \beginverse
        \[Dm]So seht, der Bauer s\"{a}t die Saat des nahen Krieges aus \[A]
        \[Dm]Kein Korn w\"{a}chst dieses Jahr f\"{u}rwahr zur Erntezeit daraus \[A]
        \[Dm]Die Ernte wird gar andrer \[G]Arten eingeholt zu \[A]dieser Zeit \[Dm]
        \[Dm]Statt golden Weizen steht ein Feld, ein schlachtenfeld ber\[G]eit \[A]
    \endverse

    \beginverse
        Und steht die M\"{u}hle, stark im Wind, dreht flei\ss{}ig Rund um Rund
        Sie gibt mit ihrem steten Knarren schwere Arbeit Kund
        Der M\"{u}ller will die gro\ss{}e Schuld an seinen Herren zahlen
        Doch wird statt feiner Feldesgaben lang schon Streit gemahlen
    \endverse

    \beginchorus
        Oh Adels\[C]mann, \echo{bedenke doch}, wer Dir einst all die \[A]Steine haute \[Dm]
        Wer dir deine Mauern \[F]baute, dieser Feste dich \[Am] umh\"{u}llt \[Dm]
        Oh Adelsmann \echo{bedneke doch}, wir dir tagein das Mahle macht
        Wer dir das Bett w\"{a}rmt jede NAcht, den Becher dir mit Weine f\"{u}llt
        Oh Adelsmann \echo{welch Nacrr du bist}, wenn Du glaubst, deine Hand sie h\"{a}lt
        Die Z\"{u}gel dieser gro\ss{}en und \[F] von Not geplagten Wel\[A]t \[Dm]
        Oh Adelsmann, ein \[F] Narr du bist, wenn Du des \[A] Volkes Macht vergisst \[Dm] \[A] \[Dm]
    \endchorus

    \beginverse
        So seht und hört den Schmiedeklang, der dröhnt nun Tag und Nacht
Der Schmied, er hat gar wochenlang sich um den Schlaf gebracht
Doch in der Esse nicht nur Erz, nein, auch sein Zorn dort siedet
Nebst Schwertern hat er gut versteckt auch Ränke dort geschmiedet
    \endverse

    \beginverse
        Und seht, die dralle Weberin ist fleißiger denn je
Sie dreht das Rad sodenn geschwind und ohne Klag und Weh
Auch wenn seit langer Weile schon das Blut ihr von den Fingern rinnt
Sie voll Inbrunst und Genuss die Fäden der Intrige spinnt
    \endverse

    \beginchorus
        \echo{Oh Adels\[C]mann}, bedenke doch, wer Dir einst all die \[A]Steine haute \[Dm]
        Wer dir deine Mauern \[F]baute, dieser Feste dich \[Am] umh\"{u}llt \[Dm]
        \echo{Oh Adelsmann}, bedneke doch, wir dir tagein das Mahle macht
        Wer dir das Bett w\"{a}rmt jede NAcht, den Becher dir mit Weine f\"{u}llt
        \echo{Oh Adelsmann}, welch Nacrr du bist, wenn Du glaubst, deine Hand sie h\"{a}lt
        Die Z\"{u}gel dieser gro\ss{}en und \[F] von Not geplagten Wel\[A]t \[Dm]
        Oh Adelsmann, ein \[F] Narr du bist, wenn Du des \[A] Volkes Macht vergisst \[Dm] \[A] \[Dm]
    \endchorus

    \beginverse
        So seht, die Knechtschaft eifert hart, um Hohen zu gefallen
        Die merken eitel nicht mal mehr, was herrscht in ihren Hallen
        Was vor sich geht, wenn kleine Meuten flüsternd über Plänen brüten
        Und ganz unverhohlen so manch dunkles Wissen hüten
    \endverse

    \beginverse
        Und seht, die Mägde streiten sich wer sich heut zum Herren legt
        Und keiner weiß, daß jede hier ein Messer unter'm Kleide trägt
        Auch in der Küche hinterm Salz ist gut manch andres Kraut versteckt
        Kein Adelsmann sollt wundern sich, wenn bald das Mahle bitter schmeckt
    \endverse
\endsong
