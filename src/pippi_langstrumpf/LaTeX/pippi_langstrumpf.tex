% vim: ft=tex:

\stepcounter{subsection}

\beginsong{Pippi Langstrumpf}[%
    by={Astrid Lindgren, Rabenschrey \textit{Medley}},
    cr={Zur Verf\"{u}gung gestellt von \textsc{Carsten vom Friedrichshain}},
    li={, adaptiert von \textsc{JoSh}},
    index={pippilangstrumpf}]

    \phantomsection
    \addcontentsline{toc}{subsection}{\thesubsection \quad \textbf{Pippi Langstrumpf} --- Astrid Lindgren, Rabenschrey \textit{Medley}}

    \label{pippilangstrumpf}

    \begin{center}
        \gtab{C}{x32010:032010}
        \gtab{Dm}{XX0231:000231}
        \gtab{G}{320033:210034}
        \gtab{F}{133411:134211}
        \gtab{Am}{X02210:002310}
    \end{center}

    \beginverse\memorize[verse]
        Zw\[C]ei mal drei macht v\[Dm]ier widde-widde-w\[G]itt und drei macht n\[C]eune,
        \[C]Ich mach' mir die W\[Dm]elt wide-wide-w\[G]ie sie mir gef\[C]\"{a}llt.
        R\[C]eit' ich im Gal\[Dm]opp holla-holla-h\[G]opsa durch die Str\[C]a\ss{}en,
        St\[C]eh'n in langen R\[Dm]eih'n alle meine Fr\[G]eunde da und schr\[C]ei'n:
    \endverse

    \beginchorus\memorize[chorus]
        \lrep H\[C]ey Pippi L\[F]angstrumpf, trallal\[G]i trallala tralla hoppsassa.
        H\[C]ey Pippi L\[F]angstrumpf, die m\[G]acht was ihr gef\[C]\"{a}llt.\rrep \rep{2}
    \endchorus

    \beginverse\replay[verse]
        Dr^ei mal drei macht s^echs, widde-widde-w^er will's von mir l^ernen?
        ^Alle gro\ss{} und kl^ein trala-lala-l^ad ich zu mir ^ein.
    \endverse

    \beginverse*\memorize[bridge]
        Ich hab' ein H\[F]aus, ein k\[G]unterbuntes H\[C]aus,
        Ein \[Am]\"{A}ffchen und ein Pf\[F]erd, die sch\[G]auen dort zum F\[C]enster raus.
        Ich hab' ein H\[F]aus, ein \[G]\"{A}ffchen und ein Pf\[C]erd
        Und j\[Am]eder der uns fr\[F]agt, kriegt' \[G]unser Ein-Mal-\[F]Eins g\[G]el\[C]ehrt.
    \endverse

    \beginverse\replay[verse]
        Zw^ei mal drei macht v^ier widde-widde-w^itt und drei macht n^eune,
        ^Ich mach' mir die W^elt wide-wide-w^ie sie mir gef^\"{a}llt.
        Dr^ei mal drei macht s^echs wide-wide-w^er will's von mir l^ernen?
        ^Alle gro\ss{} und kl^ein trala-lala-l^ad ich zu mir ^ein.
    \endverse

    \beginchorus\replay[chorus]
        \lrep H^ey Pippi L^angstrumpf trala-l^i trala-la trala hoppsassa.
        H^ey Pippi L^angstrumpf die m^acht was ihr gef^\"{a}llt. \rrep \rep{2}
    \endchorus

    \textnote{Achtung! Text! Tempo leicht anziehen!}

    \beginverse\replay[verse]
        Es g^ibt nur einen G^ott widde-widde-w^itt und der hat H^\"{o}rner.
        H^at er keine dr^ann, ja dann z^\"{u}nden wir ihn ^an.
    \endverse

    \beginchorus\replay[chorus]
        \lrep H^ey wir sind H^eiden trala-l^i trala-la trala hoppsassa.
        H^ey wir sind H^eiden, wir t^un was uns gef^\"{a}llt. \rrep \rep{2}
    \endchorus

    \beginverse\replay[verse]
        Od^in hei\ss{}t unser G^ott, widde-widde-w^itt der hat zwei R^aben.
        J^esus der war ^anders, umg^eben von zw\"{o}lf Kn^aben
    \endverse

    \beginchorus\replay[chorus]
        \lrep H^ey wir sind H^eiden trala-l^i trala-la trala hoppsassa.
        H^ey wir sind H^eiden, wir t^un was uns gef^\"{a}llt. \rrep \rep{2}
    \endchorus

    \beginverse\replay[verse]
        K^ommt ein Christ dah^er, widde-widde-w^itt dann kriegt er H^aue.
        K^ommt er noch mal ^an, ja dann ^ist er wirklich dr^an.
    \endverse

    \beginchorus\replay[chorus]
        \lrep H^ey wir sind H^eiden trala-l^i trala-la trala hoppsassa.
        H^ey wir sind H^eiden, wir t^un was uns gef^\"{a}llt. \rrep \rep{2}
    \endchorus

    \beginverse\replay[verse]
        Und k^ommt ein Weib dah^er widde-widde-w^itt oder auch zw^eie,
        N^ehmen wir sie m^it, auch wenn ^eine davon tr^itt.
    \endverse

    \beginchorus\replay[chorus]
        \lrep H^ey wir sind H^eiden trala-l^i trala-la trala hoppsassa.
        H^ey wir sind H^eiden, wir t^un was uns gef^\"{a}llt. \rrep \rep{2}
    \endchorus

    \beginverse\replay[verse]
        Der D^eutschritter ist gr^o\ss{}, mein Pf^eil der wird ihn tr^effen.
        Und tr^ifft ihn nicht mein Pf^eil, ja dann sp^\"{u}rt er halt mein B^eil.
    \endverse

    \beginchorus\replay[chorus]
        \lrep H^ey wir sind H^eiden trala-l^i trala-la trala hoppsassa.
        H^ey wir sind H^eiden, wir t^un was uns gef^\"{a}llt. \rrep \rep{2}
    \endchorus
\endsong
