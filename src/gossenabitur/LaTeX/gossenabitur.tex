\stepcounter{subsection}

\beginsong{Gossenabitur}[%
    by={Knasterbart},
    cr={},
    li={},
    index={gossenabitur}]

    \phantomsection
    \addcontentsline{toc}{subsection}{\thesubsection \quad \textbf{Gossenabitur} --- Knasterbart}

    \label{gossenabitur}

    \beginverse\memorize[verse]
        Mudder hat mir mal gesagt, ich soll was bess'res werden
        Nicht so‘n Gossenlümmel ohne Geld
        Würde ich nichts lernen, würde ich mich nur gefährden
        Und hätte meine Armut vorbestellt
        Ich solle fleißig beten, zum Schulmeister geh'n
        Und ein braver Bänkedrücker sein
        Ich soll‘ in Ärsche kriechen und um Vergebung fle'n
        Fäng‘ ich mir vom Leben eine ein
    \endverse

    \beginchorus\memorize[chorus]
        Da schraub ich mir doch lieber den Schnaps in die Figur
        Und lerne für mein Gossenabitur
        Da schraub ich mir doch lieber den Schnaps in die Figur
        Und lerne für mein Gossenabitur
    \endchorus

    \beginverse\replay[verse]
        Vadder hat mich stets ermahnt, ‘ne Arbeit zu ergreifen
        Als emsig fleiß'ger Medizinstudent
        Dazu würd‘ ich mich eignen, selbst mir Pillen zu verschreiben
        Ich wär‘ zu allen Zeiten sehr solvent
        Kein Weib täte sich zieren beim Massagen rezeptieren
        Sie würden gern zum Onkel Doktor geh'n
        Doch will ich mich nicht länger in Fantasien verlieren
        Hier in der Gosse ist es auch ganz schön
    \endverse

    \beginchorus\replay[chorus]
        Da schraub ich mir doch lieber den Schnaps in die Figur
        Und lerne für mein Gossenabitur
        Da schraub ich mir doch lieber den Schnaps in die Figur
        Und lerne für mein Gossenabitur
    \endchorus

    \beginverse\replay[verse]
        Und damit keiner sagen kann, wir ham den Dreh nicht raus
        Stell'n wir uns ein Armutszeugnis aus:
        Im „Schnorren" bin ich ausreichend, im „Läuseknacken" schlecht
        Begabt bin ich in "Straßenhehlerei"
        Ungenügend hab ich nur in "Straf- und Steuerrecht"
        Dafür in „Kneipenschlägerei" ‘ne Zwei
        In „Stammtischpolitik" bin ich der Jahrgangsbeste wohl
        Hab eine Drei in „Die Polente schmieren"
        Außerdem besitze ich das Einser-Monopol
        In „Heimlich an den Tresen urinieren"
        „Kneipenkunde" kann ich gut, ich kann mich nicht beschwer'n
        Den Leistungskurs in „Saufen" hab ich gern
        Da habe ich ne glatte Zwei im „Bierkrüge ausleer'n"
        In „Branntweinschlucken" eine Eins mit Stern
        Mein Lieblingsfach ist „Pöbeln" und „Zecheprellerei"
        Und „Gossenpoesie am Pintentresen"
        Und in „Liebeskunde" – ich sag das mal so frei:
        Da wär‘ ich gern befriedigend gewesen
        „Arbeits- und Sozialverhalten", ja was soll ich sagen
        Ja brauch man das denn wirklich hier im Leben?
        In unser'n Armutszeugnis steht, sollt jemand danach fragen
        „Sie haben sich stets Mühe gegeben!"
    \endverse

    \beginchorus\replay[chorus]
        Da schraub ich mir doch lieber den Schnaps in die Figur
        Und lerne für mein Gossenabitur
        Da schraub ich mir doch lieber den Schnaps in die Figur
        Und lerne für mein Gossenabitur
    \endchorus

    \beginchorus\replay[chorus]
        Da schraub ich mir doch lieber den Schnaps in die Figur
        Und lerne für mein Gossenabitur
        Da schraub ich mir doch lieber den Schnaps in die Figur
        Und lerne für mein Gossenabitur
    \endchorus

    \beginchorus\replay[chorus]
        Da schraub ich mir doch lieber den Schnaps in die Figur
        Und lerne für mein Gossenabitur
        Da schraub ich mir doch lieber den Schnaps in die Figur
        Und lerne für mein Gossenabitur
    \endchorus

    \beginchorus\replay[chorus]
        Da schraub ich mir doch lieber den Schnaps in die Figur
        Und lerne für mein Gossenabitur
        Da schraub ich mir doch lieber den Schnaps in die Figur
        Und lerne für mein Gossenabitur
    \endchorus
\endsong
