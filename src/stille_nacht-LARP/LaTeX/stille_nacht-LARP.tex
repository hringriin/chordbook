% vim: ft=tex:

\stepcounter{subsection}

\beginsong{Stille Nacht}[%
    by={Anja Berger},
    index={stillenacht-larp}]

    \phantomsection
    \addcontentsline{toc}{subsection}{\thesubsection \quad \textbf{Stille Nacht} --- Larp-Lieder.de}

    \label{stillenacht-larp}

    \capo{4}

    %\begin{center}
        %\gtab{C}{032010:032010}
        %\gtab{G}{320033:210034}
        %\gtab{Dm}{XX0231:000231}
        %\gtab{Am}{X02210:002310}
        %\gtab{Fm6/D}{3:XX0342:000231}
        %\gtab{Am/D}{5:XX0111:000111}
    %\end{center}

    \beginverse\memorize[verse]
        Stille Nacht, friedliche Nacht,
        alles schlaeft, keiner wacht...
        Nur die Kultisten drunten im Tal
        suchen noch Teilnehmer fuers Ritual,
        schlafe in seliger Ruhe,
        schlafe mit Frieden im Sinn.
    \endverse

    \beginverse\replay[verse]
        Stille Nacht, friedliche Nacht,
        hatte sich auch der Wandrer gedacht,
        jetzt liegt er auf dem Opferaltar,
        wo er grad eben noch Pilzsuchen war,
        schlafe in seliger Ruhe,
        schlafe mit Frieden im Sinn.
    \endverse

    \beginverse\replay[verse]
        Stille Nacht, friedliche Nacht,
        ein Messer blitzt im Dunkeln sacht
        Der Bannkreis spaeter sanft verweht,
        wo der neue Untote steht
        schlafe in seliger Ruhe,
        schlafe mit Frieden im Sinn
    \endverse

    \beginverse\replay[verse]
        Stille Nacht, friedliche Nacht,
        in der Taverne man noch lacht,
        der Frau die fast halbtot die Tuer noch erreicht,
        wird erst mal freundlich der Metkrug gereicht,
        schlafe in seliger Ruhe,
        schlafe mit Frieden im Sinn.
    \endverse

    \beginverse\replay[verse]
        Stille Nacht, friedliche Nacht,
        sei nur ruhig- gib nicht acht,
        die Schreie die dort in der Ferne zu hoern,
        koennten dich hier bloss beim trinken stoern,
        schlafe in seliger Ruhe,
        schlafe mit Frieden im Sinn.
    \endverse
\endsong
