\stepcounter{subsection}

\beginsong{Mord auf dem Abort}[%
    by={Versengold},
    index={mordaufdemabort}]

    \phantomsection
    \addcontentsline{toc}{subsection}{\thesubsection \quad Mord auf dem Abort --- Versengold}

    \label{mordaufdemabort}

    \begin{center}
        \gtab{A}{X02220:001230}
        \gtab{A}{X0222X:001110}
        \gtab{E}{022100:023100}
        \gtab{D}{XX0232:000132}
    \end{center}

    \beginverse
        Nachdem sich unser Graf denn mit dem Volke einst zerstritt
        Das wahrlich arg und schwer uznter den hohen Steuern litt
        Und voller Zorn das n\"{a}chste Mal den Eintreiber versohlte
        Der daraufhin mit S\"{o}ldnerschaft das Doppelte sich holte.

        Beschloss der Graf den Abort auf dem Bergfried neu zu richten
        Auf dass die Leute in dem Dorf den Balken konnten sichten
        Und um ihn' jeden Morgen dann f\"{u}r die geliebte Gunst zu danken
        Zeigte er mit Donnerschlag zum Morgengru\ss{} den Blanken.
    \endverse

    \beginverse*
        Nach vieler Jahre in der Schmach und lang erlebter Plage
        raunt pl\"{o}tzlich durch des Volkes Reihen schmunzelnd eine Frage:
    \endverse

    \beginchorus
        Wer hat den Donnerbalken anges\"{a}gt?
        Wen hat's zu solch gottloser Tat bewegt?
        Wer tr\"{a}gt die Handschrift von dem feigen Mord?
        An unser'm Gnaden auf dem Turm-Abort ...
    \endchorus

    \beginverse
        Nach gar nicht langer Weile kam vom k\"{o}niglichen Thron
        Bald schon eine Pfaff- und B\"{u}ttel-Aufkl\"{a}rkommission,
        Welche diese dunkle Tat f\"{u}r aller Adelswohl wollt' lichten
        Und getreulich von dem Vor- und Ab-Reinfall berichten.

        Es kamen viele Fragen auf wie die, warum der der Graben gar,
        Der um die Burg flie\ss{}t, stets gef\"{u}llt, an diesem Morgen trocken war?
        Und wer was zu den frischen Erdhaufbahnen um der Feste wei\ss{},
        Dies sei nur eine Spargelzucht bezeugte ein alt Bauerngreis
    \endverse

    \beginverse*
        Das Wasser ging wohl just zu Dunst nach Vortags Hitzeplage
        Stellte so der Hauptmann fest umsorgt von einer Frage ...
    \endverse

    \beginchorus
        Wer hat den Donnerbalken anges\"{a}gt?
        Wen hat's zu solch gottloser Tat bewegt?
        Wer tr\"{a}gt die Handschrift von dem feigen Mord?
        An unser'm Gnaden auf dem Turm-Abort ...
    \endchorus

    \beginverse
        So suchte man im Volk umher nach weit'ren Zeuzgensagen
        Und tat die h\"{o}chsten H\"{a}upter in dem Dorfrat denn befragen.
        Der Schulze war gar Augenzeuge und sprach nach 'nem lauten Krach,
        Fiel der Graf mit Rittlings-Salto und 'nem halben Auerbach

        So grazi\"{o}s von seinem Stuhl, auf dem er sa\ss{} gar nackig drauf,
        Zum Boden wo er denn noch schrie - bis ihm der Balk fiel oben auf
        Und f\"{u}gte noch hinzu, kein Wunder, dass der solcher Art verreckt,
        Der Graf hat seiner Lebzeit doch nur stets im Dung gesteckt.
    \endverse

    \beginverse*
        Die Kommission war zwar emp\"{o}rt, was hier der Schulze wage,
        Doch lie\ss{}en sie von Strafe ab, zu wichtig war die Frage.
    \endverse

    \beginchorus
        Wer hat den Donnerbalken anges\"{a}gt?
        Wen hat's zu solch gottloser Tat bewegt?
        Wer tr\"{a}gt die Handschrift von dem feigen Mord?
        An unser'm Gnaden auf dem Turm-Abort ...
    \endchorus

    \beginverse
        Auch der Wachmann wusst' nicht weiter, obgleich er am Tore stand,
        Und den werten Herrn vom Hause in der misslich Lage fand.
        Er sagte denn, er k\"{o}nnt' an sich die ganze Tat auch nicht versteh'n
            Und h\"{a}tte wohl noch nie im Leben so'n Haufen Schei\ss{} geseh'n.

        Doch dann gab er den Hinweis noch vielleicht hat's ja den Koch verleitet
        Und er hat an jenem Morgen allzu schweres Mahl bereitet.
        Der Koch doch wies emp\"{o}rt zur\"{u}ck, er k\"{O}nne nicht der T\"{a}ter sein,
        Er fl\"{o}\ss{}te uns'rem Grafen nur zwei Flaschen leichten Weines ein.
    \endverse

    \beginverse*
        Die Kommission fand's m\"{u}\ss{}ig schon und bracht's zu keiner Klage
        Und verzweifelte nur weiter and der steten Frage:
    \endverse

    \beginchorus
        Wer hat den Donnerbalken anges\"{a}gt?
        Wen hat's zu solch gottloser Tat bewegt?
        Wer tr\"{a}gt die Handschrift von dem feigen Mord?
        An unser'm Gnaden auf dem Turm-Abort ...
    \endchorus

    \beginverse
        Nach ein paar Tagen M\"{u}hsal war's den B\"{U}tteln dann zu viel
        Und auch die Pfaffen hatten schon 'n and'ren Fall zum Ziel.
        Das mag zwar wunderlich erschei'n, doch nur so lang wie ihr nicht wisst,
        Dass nun das Haupt der Komission uns neu ernanntes Gr\"{a}flein ist.

        Nachdem er denn den Abort legte auf 'ne and're Turmesseite
        Und uns von der Steuerlast zumindest einen Teil befreite
        Hernach feierte das Volk drei Tage lang voll Gl\"{u}ck und Wissen,
        dass ein jeder aus dem Dorf am Mordtag noch vom Turm geschissen.
    \endverse

    \beginverse*
        Und durch's ganze Dorf erklang nicht eine Trauerklage,
        Doch Jubelrufe auf die Antwort der gewissen Frage.
    \endverse

    \beginchorus
        Ich hab den Donnerbalken anges\"{a}gt,
        Mich hat's zu solch gottloser Tat bewegt,
        Mein ist die Handschrift von dem feigen Mord
        An unser'm Gnaden auf dem Turm-Abort

        Wir ham' den Donnerbalken anges\"{a}gt,
        Uns hat's zu solch gottloser Tat bewegt,
        Uns're ist die Handschrift von dem feigen Mord,
        An unser'm Gnaden auf dem Turm-Abort.
    \endchorus
\endsong
